\chapter{Теория управления и\\[-20pt] \hfill эксперные системы}

\AddToShipoutPictureBG*{%
  \AtPageUpperLeft{%
    \hspace{\paperwidth}%
    \raisebox{-\baselineskip}{%
      \makebox[-10pt][r]{\textbf{ТИИ}}
}}}%

\begin{multicols}{2}
    \raggedcolumns
    \section{Собственные числа и собственные векторы матриц.}
    \section{Квадратичные формы. Свойства положительно полуопределенных и положительно
    определенных матриц.}
    \section{Теорема о существовании и единственности решения задачи Коши для системы
    обыкновенных дифференциальных уравнений.}
    \section{Устойчивость по Ляпунову динамических систем.}
    \section{Логика исчисления предикатов первого порядка. Дайте определения понятиям терм,
    предикат, формула. Перечислите основные отличия логики предикатов от логики
    высказываний. Синтаксис и семантика языка первого порядка. Примеры.}
    \section{Преобразование Фурье. Понятие о прямом и обратном преобразовании Фурье. Свойства
    преобразования Фурье.}
    \section{Вейвлет-преобразование и анализ временных рядов. Непрерывное, дискретное и
    быстрое вейвлет-преобразования. Основные понятия теории принятия решений, задача
    принятия решения, процесс принятия решения. Способы оценки, сравнения и выбора
    варианта. Описание подходов к решению задач коллективного выбора.}
    \columnbreak
    \section{Математические методы в экспертных системах. Основные компоненты экспертных
    систем. Этапы разработки. Инструменты разработки. Инженерия знаний.}
\end{multicols}