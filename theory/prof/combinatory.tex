\chapter{Комбинаторика}

\AddToShipoutPictureBG*{%
  \AtPageUpperLeft{%
    \hspace{\paperwidth}%
    \raisebox{-\baselineskip}{%
      \makebox[-10pt][r]{\textbf{ПМИ}, ИВТ, БИ}
}}}%

\begin{multicols}{2}
    \raggedcolumns
    \section{Основные правила комбинаторики: правило сложения, умножения. Основные
    комбинаторные объекты: сочетания и размещения с повторениями и без повторений.
    Формулы для количества сочетаний и размещений. Принцип Дирихле. Примеры.}
      \subsection*{Правило сложения}

      \begin{proposition}{}{}
        Пусть у нас есть множество $A$, содержащее $n$ объектов, и $B$, содержащее $m$ объектов. Тогда число способов выбрать 1 объект из $A$ \textbf{или} один объект из $B$ равно $n + m$.
      \end{proposition}

      \subsection*{Правило умножения}

      \begin{proposition}{}{}
        Пусть у нас есть множество $A$, содержащее $n$ объектов, и $B$, содержащее $m$ объектов. Тогда число способов выбрать 1 объект из $A$ \textbf{и} один объект из $B$ равно $n \cdot m$.
      \end{proposition}

      \subsection*{Способы выбора объектов из множества}

      \subsubsection*{Размещения с повторениями}

      \begin{definition}{}{}
        Числом $\bar{A}_n^k$ называется количество способов выбрать $k$ элементов из множества $n$ элементов так, что при этом нам \textbf{важен} порядок выбора и мы \textbf{допускаем} повторения элементов.
      \end{definition}

      \begin{note}{}{}
        Читается как <<$A$ из $n$ по $k$ с чертой>>.
      \end{note}

      \begin{note}{}{}
        Размещение из $k$ элементов с повторениями также называют \textit{$k$-размещением с повторениями}.
      \end{note}

      \begin{theorema}{}{}
        \[
          \bar{A}_n^k = n^k.
        \]
      \end{theorema}

      \subsubsection*{Размещения без повторений}

      \begin{definition}{}{}
        \textit{Факториалом} числа $n > 0$ называют число
        \[
          n! = n \cdot (n - 1) \cdot \ldots \cdot 2 \cdot 1.
        \]
        При этом считают, что
        \[
          0! = 1! = 1.
        \]
      \end{definition}

      \begin{definition}{}{}
        Числом $A_n^k$ называется количество способов выбрать $k$ элементов из множества $n$ элементов так, что при этом нам \textbf{важен} порядок выбора и мы \textbf{не допускаем} повторения элементов.
      \end{definition}

      \begin{note}{}{}
        Читается как <<$A$ из $n$ по $k$>>.
      \end{note}

      \begin{note}{}{}
        Размещение из $k$ элементов без повторений также называют \textit{$k$-размещением без повторений}.
        При этом $n$-размещение без повторений называется \textit{перестановкой}.
      \end{note}

      \begin{theorema}{}{}
        \[
          A_n^k = n \cdot (n - 1) \cdot \ldots \cdot (n - k + 1) = \frac{n!}{(n - k)!}.
        \]
      \end{theorema}

      \subsubsection*{Сочетания без повторений}

      \begin{definition}{}{}
        Множество элементов, из которых составлен объект (то же множество, перестановка или размещение), называется \textit{набором} или же \textit{сочетанием}.
      \end{definition}

      \begin{definition}{}{}
        Числом $C_n^k$ называется количество способов выбрать $k$ элементов из множества $n$ элементов так, что при этом нам \textbf{не важен} порядок выбора и мы \textbf{не допускаем} повторения элементов. (То есть количество различных наборов размера $k$, которые можно получить из $n$ элементного множества).
      \end{definition}

      \begin{note}{}{}
        Читается как <<$C$ из $n$ по $k$>>.
      \end{note}

      \begin{note}{}{}
        Сочетание из $k$ элементов без повторений также называют \textit{$k$-сочетанием без повторений}.
      \end{note}

      \begin{theorema}{}{}
        \[
          C_n^k = \frac{A_n^k}{k!} = \frac{n!}{k! (n - k)!}.
        \]
      \end{theorema}

      \subsubsection*{Сочетания с повторениями}

      \begin{definition}{}{}
        Числом $\bar{C}_n^k$ называется количество способов выбрать $k$ элементов из множества $n$ элементов так, что при этом нам \textbf{не важен} порядок выбора и мы \textbf{допускаем} повторения элементов. (Количество различных $k$-сочетаний с повторениями).
      \end{definition}

      \begin{theorema}{}{}
        \[
          \bar{C_n^k} = C_{n + k - 1}^k.
        \]
      \end{theorema}

      \begin{proof}
        Поймём, что любой набор с повторениями определяется числами вхождений каждого элемента в данный набор. Для определённости будем считать, что мы работаем с множеством $A$:
        \[
          A = \{a_1, \dots, a_n\}.
        \]
        Обозначим как $u_i$~---~число вхождений $a_i$ в данный набор. Тогда сразу следует равенство
        \[
          u_1 + \ldots + u_n = k.
        \]
        Теперь построим последовательность нулей и единиц, которая однозначно задаст нам набор:
        \[
          \underbrace{1 \ldots 1}_{u_1 \text{ раз}}
          0
          \underbrace{1 \ldots 1}_{u_2 \text{ раз}}
          0 \ldots 0
          \underbrace{1 \ldots 1}_{u_n \text{ раз}}.
        \]
        То есть мы записываем между нулями-разделителями столько единиц, сколько у нас имеется $a_i$ в наборе.
        \begin{itemize}
          \item Сколько всего нулей? Ответ: $n - 1$.
          \item Сколько всего единиц? Ответ: $k$.
          \item Какова длина всей последовательности? Ответ: $n + k - 1$.
        \end{itemize}
        При этом длина последовательности всегда одинакова. Давайте просто выберем $k$ позиций среди $n + k - 1$ в ней, куда мы поставим единицы, а в остальных местах будут нули. Тогда мы получим какую-то последовательность, которая точно описывает какой-то из наборов. Отсюда
        \[
          \bar{C}_n^k = C_{n + k - 1}^k.
        \]
      \end{proof}

      \subsection*{Принцип Дирихле}

      \begin{definition}{}{}
        Пусть у нас есть $n + 1$ кролик и $n$ клеток для них. Тогда абсолютно очевидно, что если мы заполним все клетки, то в одной из них будет 2 кролика. Это и называется \textit{принципом Дирихле}. Более формально ещё можно сказать так:
        
        \textit{
        Если у нас есть $nk + 1$ объект и $n$ ящиков, то в каком-нибудь ящике окажется не менее $k + 1$ объектов.
        }
      \end{definition}

      \Ex У нас есть квадрат со стороной 2. Если мы выберем 5 произвольных точек на его границах или внутри него, то хотя бы 2 из них будут на расстоянии не более $\sqrt{2}$.

      \begin{proof}
        Разделим квадрат на 4 меньших со стороной 1. Тогда, по принципу Дирихле хотя бы в одном из таких квадратов будет 2 точки, а наибольшее расстояние между точками в квадрате~---~это его диагональ, то есть $\sqrt{1^2 + 1^2} = \sqrt{2}$
      \end{proof}

    \section{Множества. Круги Эйлера, операции на множествах. Формула включений и
    исключений. Примеры.}
    \subsection*{Формула включений и исключений}

    \subsection*{Множества}
 
    \begin{definition}{}{}
        \textit{Множеством} называется совокупность каких-либо объектов.
    \end{definition}
     
    \begin{note}{}{}
        Если говорить чуть точнее, то множество считается неопределяемым понятием, так как его определение даётся через синонимичные слова, что является замкнутым кругом.
    \end{note}
     
    \begin{note}{}{}
        В рамках стандартной модели (так мы будем называть наивную теорию множеств, ибо будем работать по большому счёту только с ней) объектом может быть что угодно, в~том числе и множество.
    \end{note}
     
    \subsubsection*{Свойства множества}
     
    \begin{enumerate}
         \item Каждый объект входит в множество ровно один раз, то есть множество хранит \textit{уникальные} объекты. Если мы рассматриваем множества без этого свойства, то они называются \textit{мультимножествами}.
         \item Множество не обладает порядком. Если порядок объектов в множестве важен, то такое множество называется \textit{кортежом}, или же \textit{упорядоченным множеством}.
         \item Запись $x \in Y$ означает, что объект $x$ принадлежит множеству $Y$.
    \end{enumerate}
    
    \begin{note}{}{}
      Еще вводится обозначение $x \notin Y$, что по определению означает $\neg (x \in Y)$.
    \end{note}
     
    \begin{definition}{}{}
        \textit{Элементом} множества называется объект, который принадлежит этому множеству.
    \end{definition}
     
    \begin{definition}{}{}
        Выражение $X \subset Y$ означает, что множество $X$ является \textit{подмножеством} $Y$. Формально записывается так:
        $$
            X \subset Y \Ra (\forall x \in X \Ra x \in Y).
        $$
    \end{definition}
     
    \subsection*{Равенство множеств}
     
    \begin{definition}{}{}
        $X = Y$, если $(X \subset Y) \wedge (Y \subset X)$, или же $\forall z \in X \Ra z \in Y$ и $\forall z \in Y \Ra z \in X$.
    \end{definition}
     
    \subsubsection*{Свойства равенства множеств}
     
    \begin{enumerate}
         \item $X \subset X$ (рефлексивность).
         \item $(X \subset Y) \wedge (Y \subset X) \Ra (X = Y)$ (антисимметричность).
         \item $(X \subset Y) \wedge (Y \subset Z) \Ra (X \subset Z)$ (транзитивность).
    \end{enumerate}
     
    \begin{note}{}{}
        При этом стоит отметить, что принадлежность не обладает транзитивностью. Контрпримером служит выражение:
        $$
            1 \in \{1\} \in \{2, 3, \{1\}\},
        $$
        но при этом $1 \notin \{2, 3, \{1\}\}$.
    \end{note}
     
    \subsection*{Способы описания множеств}
     
    \begin{enumerate}
         \item Прямое перечисление элементов: $X = \{1, 2, 3\}$.
         \item Генератор множества (set builder notation) $X = \{x\ |\ x = 2k, k \in \N\}$.
    \end{enumerate}
     
    \subsection*{Парадокс Рассела}
     
    Если множество может быть элементом множества, то существует ли множество всех множеств? С этим вопросом мы приходим быстро к противоречию.
     
    \begin{proposition}{}{}
        Рассмотрим $M = \{x\ |\ x \notin x\}$. Верно ли утверждение $M \in M$?
    \end{proposition}
     
    \begin{proof}
        Имеем 2 случая:
        
        \begin{enumerate}
            \item $M \in M$. Но из определения $\forall x \in M \Ra x \notin x$, мы получаем противоречие.
            \item $M \notin M$. Но из определения $x \notin x \Ra x \in M$. Снова противоречие.
        \end{enumerate}
        
        Таким образом, множества всех множеств не существует.
    \end{proof}
     
    \begin{note}{}{}
        Выше мы говорили о равенстве множеств. Как известно, мы определяем равенство как отношение эквивалентности на некотором множестве, но так как мы показали, что множества всех множеств не существует, то мы не можем назвать равенство между множествами отношением эквивалентности.
    \end{note}
     
    \subsection*{Пустое множество}
     
    \begin{definition}{}{}
        \textit{Пустым множеством} называется такое множество, в котором нету элементов. Обозначается обычно так: $\emptyset$.
    \end{definition}
     
    \subsubsection*{Свойства пустого множества}
     
    \begin{enumerate}
         \item Пустое множество единственно.
         \item Пустое множество вложено в любое другое множество: $\forall X \Ra \emptyset \subset X$.
    \end{enumerate}
     
    \subsection*{Различие между принадлежностью и подмножеством}
     
    Рассмотрим $X$ - некоторое конечное множество, содержащее $n$ элементов.
     
    Сколько подмножеств у такого множества? Ответ: $2^n$.
     
    \begin{itemize}
         \item $n = 0 \Ra \emptyset$~---~1 подмножество, 0 элементов.
         \item $n = 1 \Ra \emptyset, \{a\}$~---~2 подмножества, 1 элемент.
         \item $n = 2 \Ra \emptyset, \{a\}, \{b\}, \{a, b\}$~---~4 подмножества, 2 элемента.
    \end{itemize}
     
    \subsection*{Универсальное множество}
     
    \begin{definition}{}{}
        \textit{Универсальным множеством} называется такое множество, для которого в конкретно задаче считается верным, что для любого множества $A$ выполнены два свойства:
        \begin{itemize}
            \item $A \cap U = A$,
            \item $A \cup U = U$.
        \end{itemize}
    \end{definition}
     
    \subsection*{Операции над множествами}
     
    \begin{enumerate}
         \item Объединение: $A \cup B = \{x\ |\ (x \in A) \vee (x \in B)\}$.
         \item Пересечение: $A \cap B = \{x\ |\ (x \in A) \wedge (x \in B)\}$.
         \item Разность: $A \backslash B = \{x\ |\ (x \in A) \wedge (x \notin B)\}$.
         \item Симметрическая разность: $A \triangle B = \{x\ |\ (x \in A \cup B) \wedge (x \notin A \cap B)\}$.
         \item Отрицание (дополнение): $\overline{A} = \{x\ |\ x \notin A\}$.
    \end{enumerate}
    
    \subsubsection*{Дистрибутивность}
    
    \begin{proposition}{}{}
      Для любых множеств $A, B$ и $C$ верно, что
      $$
        A \cap (B \cup C) = (A \cap B) \cup (A \cap C).
      $$
    \end{proposition}
    
    \begin{proof}
      Пусть $x \in A \cap (B \cup C)$, тогда:
      $$
      (x \in A) \wedge (x \in (B \cup C)).
      $$
      Имеем 2 случая:
      \begin{enumerate}
        \item $(x \in A) \wedge (x \in B) \Ra x \in (A \cap B)$.
        \item $(x \in A) \wedge (x \in C) \Ra x \in (A \cap C)$.
      \end{enumerate}
      Фактически означает, что $x \in (A \cap B) \cup (A \cap C) \Ra A \cap (B \cup C) \subset (A \cap B) \cup (A \cap C)$.
      Теперь покажем обратное. Рассмотрим $x \in (A \cap B) \cup (A \cap C)$.
      Возникает снова 2 случая:
      \begin{enumerate}
        \item $x \in (A \cap B) \Ra (x \in A) \wedge (x \in B) \Ra (x \in A) \wedge (x \in (B \cup C)) \Ra x \in A \cap (B \cup C)$.
        \item $x \in (A \cap C) \Ra (x \in A) \wedge (x \in C) \Ra (x \in A) \wedge (x \in (B \cup C)) \Ra x \in A \cap (B \cup C)$.
      \end{enumerate}
    
      Отсюда по определению равенства $A \cap (B \cup C) = (A \cap B) \cup (A \cap C)$.
    \end{proof}
    
    \subsubsection*{Законы де Моргана}
    
    Законы де Моргана на множествах имеют ровно такие же аналоги, как и в логике:
    \begin{align*}
      \overline{A \cap B} = \overline{A} \cup \overline{B},
      \\
      \overline{A \cup B} = \overline{A} \cap \overline{B}.
    \end{align*}
    
    \subsection*{Упорядоченные пары и кортежи}
    
    \begin{definition}{}{}
      \textit{Неупорядоченной парой} называется мультимножество из 2х элементов. Обозначается как и просто множество: $\{a, b\}$.
    \end{definition}
    
    \begin{definition}{}{}
      \textit{Упорядоченной парой} называется неупорядоченная пара, у которой зафиксирован первый элемент. Обозначается через круглые скобки: $(a, b)$. Упорядоченная пара может быть выражена через мультимножество по определению Куратовского:
      \begin{itemize}
        \item упрощенное определение Куратовского $\{a, \{a, b\}\}$,
        \item полное определение Куратовского $\{\{a\}, \{a, b\}\}$.
      \end{itemize}
    \end{definition}

Пусть есть $N$ элементов. Обозначим $N(\alpha_i)$ --- количество элементов, обладающих свойством $\alpha_i$. $N(\alpha'_i)$ --- количество элементов, не обладающих свойством $\alpha_i$. Ну и понятно, что $N(\alpha_i, \alpha'_j)$ --- количество элементов, обладающих свойством $\alpha_i$ \textbf{и} не обладающих свойством $\alpha_j$.

\begin{theorema}{}{}
	Если мы рассмотрим $n$ свойств, которые мы можем приписать $N$ объектам, то имеет место \textit{формула включений и исключений}:
	\begin{multline*}
		N(\alpha'_1, \ldots, \alpha'_n) = N - N(\alpha_1) - \ldots - N(\alpha_n) + \\ +
		N(\alpha_1, \alpha_2) + \ldots + N(\alpha_{n - 1}, \alpha_n) - \ldots + (-1)^n N(\alpha_1, \ldots, \alpha_n).
	\end{multline*}
\end{theorema}

\begin{proof}
	Воспользуемся математической индукцией
	\begin{itemize}
		\item База $n = 1$:
		\[
			N(\alpha'_1) = N - N(\alpha_1)
		\]
		верность очевидна.
		
		\item Предположение индукции: формула включений и исключений верна \textbf{для любых $N$ объектов и для любых $n$ свойств}. Докажем, что она также верна и в случае $(n + 1)$-го свойства для данного $N$.
		
		Применим предположение индукции для $N$ объектов и свойствам $\alpha_1, \ldots, \alpha_n$:
		\begin{multline*}
			N(\alpha'_1, \ldots, \alpha'_n) = N - N(\alpha_1) - \ldots - N(\alpha_n) + \\ +
			N(\alpha_1, \alpha_2) + \ldots + N(\alpha_{n - 1}, \alpha_n) - \ldots + (-1)^n N(\alpha_1, \ldots, \alpha_n).
		\end{multline*}
		Теперь сделаем то же самое для $M \le N$ объектов, которые точно обладают свойством $\alpha_{n + 1}$, и свойств $\alpha_1, \ldots, \alpha_n$.
		\begin{multline*}
			M(\alpha'_1, \ldots, \alpha'_n) = M - M(\alpha_1) - \ldots - M(\alpha_n) + \\
			M(\alpha_1, \alpha_2) + \ldots + M(\alpha_{n - 1}, \alpha_n) - \ldots + (-1)^n M(\alpha_1, \ldots, \alpha_n)
		\end{multline*}
		В силу определения $M$ также верно, что $M := N(\alpha_{n + 1})$. То есть можно переписать последнее выражение в виде
		\begin{multline*}
			N(\alpha'_1, \ldots, \alpha'_n, \alpha_{n + 1}) = N(\alpha_{n + 1}) - N(\alpha_1, \alpha_{n + 1}) - \ldots - N(\alpha_n, \alpha_{n + 1}) + \\
			N(\alpha_1, \alpha_2, \alpha_{n + 1}) + \ldots + N(\alpha_{n - 1}, \alpha_n, \alpha_{n + 1}) - \ldots + (-1)^n N(\alpha_1, \ldots, \alpha_n, \alpha_{n + 1}).
		\end{multline*}
		Теперь вычтем полученное выражение из того, что было для всех $N$ объектов:
		\begin{multline*}
			N(\alpha'_1, \ldots, \alpha'_n, \alpha'_{n + 1}) = N(\alpha'_1, \ldots, \alpha'_n) - N(\alpha'_1, \ldots, \alpha'_n, \alpha'_{n + 1}) = \\ =
			N - N(\alpha_1) - \ldots - N(\alpha_n) - N(\alpha_{n + 1}) + \\ +
			N(\alpha_1, \alpha_2) + \ldots + N(\alpha_n, \alpha_{n + 1}) - \ldots + (-1)^n N(\alpha_1, \ldots, \alpha_n, \alpha_{n + 1}).
		\end{multline*}
	\end{itemize}
\end{proof}

\begin{note}{}{}
	По понятным причинам слагаемых, содержащих ровно $k$ свойств, будет $C_n^k$.
\end{note}

	\cons Пусть у нас есть множество $A = \{a_1, \ldots, a_n\}$. Рассмотрим все возможные $m$-размещения с повторениями из этого множества, при этом $m < n$. их $N := n^m$ штук. Положим их объектами для формулы включений и исключений и скажем, что $N(\alpha_i)$~---~это все размещения, в которые \textbf{не} входит элемент $a_i$. Тогда, верны следующие утверждения:
	\begin{align*}
		&N(\alpha_i) = (n - 1)^m,
		\\
		&N(\alpha_i, \alpha_j) = (n - 2)^m,
		\\
		&N(\alpha_1, \ldots, \alpha_n) = (n - n)^m = 0,
		\\
		&N(\alpha'_1, \ldots, \alpha'_n) = 0, \text{ так как } m < n.
	\end{align*}
	По формуле включений и исключений имеем:
	\[
		\suml_{k = 0}^n (-1)^k \cdot C_n^k \cdot (n - k)^m = 0.
	\]
    \section{Сочетания. Размещения, перестановки и сочетания. Бином Ньютона. Треугольник
    Паскаля. Сочетания с повторениями.}
    \subsection*{Бином Ньютона}

    \begin{definition}{}{}
      \textit{Биномом Ньютона} называется выражение
      \[
        (a + b)^n = \suml_{k = 0}^n C_n^k a^k b^{n - k}.
      \]
    \end{definition}
    
    \begin{proof}
      $n$-ю степень суммы можно записать в виде
      \[
        (a + b)^n = \underbrace{(a + b) \cdot (a + b) \cdot \ldots \cdot (a + b)}_{n \text{ раз}}.
      \]
      Чтобы получить слагаемое в сумме, мы должны последовательно выбрать из каждой скобки $a$ или $b$. Любое слагаемое точно будет иметь вид
      \[
        a^k \cdot b^{n - k}.
      \]
      Более того, чтобы определить слагаемое, нам необходимо и достаточно знать, сколько $a$ мы выбрали. При этом выбирать его можно в любых скобках, а это можно сделать $C_n^k$ способами. Отсюда и формула
      \[
        (a + b)^n = \suml_{k = 0}^n C_n^k a^k b^{n - k}.
      \]
    \end{proof}
    
    \subsection*{Свойства биномиальных коэффициентов}
    
    \begin{theorema}{}{}~
      \begin{enumerate}
        \item $C_n^k = C_n^{n - k}$;
        \item $C_n^k = C_{n - 1}^k + C_{n - 1}^{k - 1}$;
        \item $C_n^0 + C_n^1 + C_n^2 + \ldots + C_n^n = 2^n$;
        \item $\left(C_n^0\right)^2 + \ldots + \left(C_n^n\right)^2 = C_{2n}^n$;
        \item $C_{n + m - 1}^{n - 1} + \ldots + C_{n - 1}^{n - 1} = C_{n + m}^n = C_{n + m}^m,\ m \ge 0$.
      \end{enumerate}
    \end{theorema}
    
    \begin{proof}~
      \begin{enumerate}
        \item Выбрать $k$ объектов из $n$ - это то же самое, что оставить $n - k$ объектов из $n$.
        
        \item Количество способов выбрать $k$-набор из $n$ элементного множества уже известно: $C_n^k$. Заметим, что каждый набор либо содержит $n$-й элемент, либо нет. То есть все наборы можно разбить на 2 группы:
        \begin{enumerate}
          \item Все наборы, которые содержат $n$-й элемент. Помимо него в них ещё надо выбрать $k - 1$ элемент, а стало быть, их всего $C_{n - 1}^{k - 1}$ штук.
          
          \item Все наборы, которые не содержат $n$-й элемент. То есть набор выбирается только из первых $n - 1$ элементов. Отсюда их $C_{n - 1}^k$ штук.
        \end{enumerate}
        Так как эти две группы в сумме составляют все возможные наборы, то и очевиден ответ
        \[
          C_n^k = C_{n - 1}^k + C_{n - 1}^{k - 1}.
        \]
        
        \item Сколько существует подмножеств у множества $n$ элементов? --- $2^n$. С другой стороны, каждое из этих подмножеств характеризуется своей мощностью, а число подмножеств, чья мощность $i$, равно $C_n^i$. Отсюда
        \[
          \suml_{i = 0}^n C_n^i = 2^n.
        \]
        
        \item Рассмотрим всевозможные $n$-наборы из $2n$ элементного множества. Их $C_{2n}^n$ штук. С другой стороны, пусть $i \in [0, \ldots, n]$ --- количество элементов для набора, которые мы возьмём из первой части исходного множества. Тогда из второй части мы выберем $n - i$ элементов. В итоге, получим сумму
        \[
          \suml_{i = 0}^n C_n^i \cdot C_n^{n - i} = \suml_{i = 0}^n \left(C_n^i\right)^2 = C_{2n}^n.
        \]
        
        \item Давайте рассмотрим всевозможные $m$-сочетания с повторениями в множестве $A = \{a_1, \ldots, a_{n + 1}\}$. Их $\bar{C}_{n + 1}^m = C_{n + 1 + m - 1}^m = C_{n + m}^m = C_{n + m}^n$ штук. С другой стороны, каждое сочетание принадлежит группе наборов, которые содержат $i \in [0, \ldots, m]$ элементов $a_1$. Отсюда
        \[
          \begin{array}{c}
            C_{n + m}^n = \suml_{i = 0}^m \bar{C}_n^{m - i} = \suml_{i = 0}^m C_{n + m - i - 1}^{m - i} = \\ = \suml_{i = 0}^m C_{n + m - i - 1}^{(n + m - i - 1) - (m - i)} = \suml_{i = 0}^m C_{n + m - 1 - i}^{n - 1}.            
          \end{array}
        \]
      \end{enumerate}
    \end{proof}
    \section{Графы: неориентированные, ориентированные, простые графы, мультиграфы и
    псевдографы. Изоморфизм графов. Некоторые стандартные классы графов: полные,
    двудольные, цепи, циклы, деревья. Критерий двудольности графа.}
    \begin{definition}{}{}
      \textit{Графом} называется пара \textit{множества вершин} и \textit{множества рёбер}.
    \end{definition}
    
    \begin{definition}{}{}
      Граф $G = (V, E)$ называется \textit{обыкновенным (простым)}, если выполнены следующие условия:
      \begin{enumerate}
        \item Нет <<петель>>, то есть
        \[
          \forall x \in V\ \ \not\exists (x, x) \in E
        \]
        
        \item Нет ориентации, то есть
        \[
          \forall x, y \in V\ \ (x, y) = (y, x)
        \]
        
        \item Нет кратных рёбер, то есть
        \[
          E \subseteq C_V^2
        \]
      \end{enumerate}
    \end{definition}
    
    \begin{note}{}{}
      $C_V^2$ обозначает множество пар вершин из $V$ без повторений наборов в них.
    \end{note}
    
    \begin{note}{}{}
      В дальнейшем, если мы говорим о графе без каких-либо оговорок, то подразумевается именно простой граф.
    \end{note}

    \begin{definition}{}{}
      Если мы отказываемся в определении обыкновенного графа от \textbf{первого} свойства, то он называется \textit{псевдографом}.
    \end{definition}
    
    \begin{definition}{}{}
      Если мы отказываемся в определении обыкновенного графа от \textbf{второго} свойства, то он называется \textit{орграфом (ориентированным графом)}.
    \end{definition}
    
    \begin{definition}{}{}
      Если мы отказываемся в определении обыкновенного графа от \textbf{третьего} свойства, то он называется \textit{мультиграфом} (не путать с гиперграфом!!!).
    \end{definition}
    
    \begin{note}{}{}
      Естественно, определения можно комбинировать.
    \end{note}
    
    \begin{definition}{}{}
      Граф $K_n = (V, E),\ |V| = n$ называется \textit{полным}, если у него есть все возможные рёбра. То есть $|E| = C_n^2$
    \end{definition}
    
    \Ex      Сколько существует графов на $V = \{1, \ldots, n\}$ вершинах?
      
      У нас $C_n^2$ рёбер и каждое мы можем либо включить, либо не брать в наш граф. Отсюда их $2^{C_n^2}$ штук.
    
    \begin{definition}{}{}
      Графы $G_1 = (V_1, E_1)$ и $G_2 = (V_2, E_2)$ называются \textit{изоморфными}, если существует биекция, удовлетворяющая следюущему условию:
      \[
        \phi \colon V_1 \to V_2 :\  \forall e = (x, y)\ \ e \in E_1 \lra (\phi(x), \phi(y)) \in E_2
      \]
    \end{definition}
    
    \begin{definition}{}{}
      \textit{Степенью вершины} $v \in V$ называется количество рёбер, инцидентных ей. Обозначается как $\deg v$
    \end{definition}
    
    \begin{definition}{}{}
      \textit{Входящей степенью} $\indeg v$ называется число рёбер, инцидентных данной вершине, в которых $v$ стоит на \textbf{втором} месте:
      \[
        \indeg v = |\{y: \  (y, v) \in E\}|
      \]
    \end{definition}
    
    \begin{definition}{}{}
      \textit{Исходящей степенью} $\outdeg v$ называется число рёбер, инцидентных данной вершине, в которых $v$ стоит на \textbf{первом} месте:
      \[
        \outdeg v = |\{y :\  (v, y) \in E\}|
      \]
    \end{definition}
    
    \begin{note}{}{}
      Для простого графа $\forall v \in V\ \indeg v = \outdeg v$. Определения, данные выше, получают смысл для орграфов.
    \end{note}
    
    \begin{definition}{}{}
      Говорят, что в графе вершина $v \in V$ \textit{инцидентна} ребру $e \in E$ (или ребро $e$ инцидентно вершине $v$), если $e$ содержит в себе эту вершину.
    \end{definition}
    
    \begin{lemma}{(О рукопожатиях)}{}
      В графе любого типа $G = (V, E)$ верно утверждение:
      \[
        \suml_{v \in V} \deg v = 2|E|
      \]
    \end{lemma}
    
    \begin{proof}
      Давайте мысленно зафиксируемся на каком-то из рёбер, и начнём суммировать степени вершин. Наше ребро может быть учтено лишь тогда, когда мы будем считать вершины ему инцидентные, коих всего 2. Отсюда и получается равенство.
    \end{proof}
    
    \begin{definition}{}{}
      Граф называется \textit{регулярным}, если степени всех вершин одинаковы.
    \end{definition}
    
    \begin{definition}{}{}
      \textit{Маршрутом в графе} $G = (V, E)$ будем называть чередующуюся последовательность вершин и рёбер, которая начинается и заканчивается на вершинах.
      \[
        v_1 e_1 v_2 e_2 \ldots e_n v_{n + 1}, \quad \forall e_i\ \ e_i = (v_i, v_{i + 1})
      \]
    \end{definition}
    
    \begin{note}{}{}
      Определение маршрута, естественно, допускает возможность появления одинаковых рёбер и вершин в последовательности.
    \end{note}
    
    \begin{definition}{}{}
      Маршрут называется \textit{замкнутым}, если $v_1 = v_{n + 1}$
    \end{definition}
    
    \begin{definition}{}{}
      Если в замкнутом маршруте все рёбра разные, то он называется \textit{циклом}
    \end{definition}
    
    \begin{lemma}{}{}
      В замкнутом маршруте можно найти цикл. То есть из замкнутого маршрута получить корректный маршрут цикла.
    \end{lemma}
    
    \begin{proof}
      Пусть есть замкнутый маршрут. Возможно 2 ситуации:
      \begin{enumerate}
        \item Все рёбра замкнутого маршрута оказались разными. Тогда он будет циклом по определению.
        
        \item Нашлось хотя бы 2 одинаковых ребра. В силу конечности маршрута, мы можем рассмотреть такие 2 одинаковых ребра $e = (v, u)$, что между ними в маршруте стоят только разные рёбра. Возможно снова 2 ситуации:
        \begin{enumerate}
          \item В маршруте по ребру $e$ мы прошли с разных сторон.
          
          Если мы прошли $v \to u$, а потом $u \to v$, то мы нашли цикл, у которого начало и конец будут на вершине $u$. Как маршрут это бы означало следующее:
          \[
            veu \ldots uev \mapsto u \ldots u
          \]
          
          \item В маршруте по ребру $e$ мы прошли с одной и той же стороны.
          
          То есть вначале могло быть $u \to v$, но и потом вышло так же $u \to v$. Снова нашли цикл, который начинается и заканчивается в вершине $u$.
          \[
            uev \ldots uev \mapsto uev \ldots u
          \]
        \end{enumerate}
      \end{enumerate}
    \end{proof}
    
    \begin{definition}{}{}
      Цикл называется \textit{простым}, если помимо разных рёбер, у него все промежуточные вершины тоже разные (то есть кроме $v_1$ и $v_{n + 1}$).
    \end{definition}
    
    \begin{definition}{}{}
      Если маршрут не замкнут и все его рёбра разные, то он называется \textit{путём (или же цепью)}.
    \end{definition}
    
    \begin{definition}{}{}
      Путь называется \textit{простым}, если все его вершины разные.
    \end{definition}
    
    \begin{definition}{}{}
      Граф $G = (V, E)$ \textit{связен}, если для любой пары вершин $x, y \in V$ существует маршрут, начинающийся в $x$ и заканчивающийся в $y$.
    \end{definition}
    
    \begin{note}{}{}
      Есть между двумя вершинами в графе есть маршрут, то есть и простой путь. Действительно, давайте как-нибудь <<выкинем>> из маршрута части между двумя одинаковыми промежуточными вершинами так, чтобы больше одинаковых промежуточных вершин не осталось.
    \end{note}
    
    \begin{note}{}{}
      Отношение существования пути между вершинами является отношением эквивалентности на множестве $V$.
    \end{note}
    
    \begin{definition}{}{}
      Граф называется \textit{ациклическим}, если в нём не содержится циклов.
    \end{definition}
    
    \begin{definition}{}{}
      Граф $G = (V, E)$ называется \textit{деревом}, если он является связным ациклическим графом.
    \end{definition}
    
    \begin{theorema}{}{}
      Для графа $G = (V, E)$ следующие 4 утверждения эквивалентны:
      \begin{enumerate}
        \item $G$ - дерево
        
        \item В $G$ любые 2 вершины соединены единственной простой цепью
        
        \item $G$ связен и если $|V| = n$, то $|E| = n - 1$
        
        \item $G$ ацикличен и если $|V| = n$, то $|E| = n - 1$
      \end{enumerate}
    \end{theorema}
    
    \begin{proof}
      Построим цикл утверждений:
      \begin{itemize}
        \item $1 \Ra 2$ В силу определения дерева $G$, между двумя вершинами будет существовать простой путь. Если их как минимум 2, то на них можно найти цикл, что противоречит ацикличности дерева.
        
        \item $2 \Ra 3$ Связность очевидна. Для доказательства второго факта, воспользуемся индукцией по $n$:
        \begin{itemize}
          \item База $n = 1$ тривиальна.
          
          \item Переход $n > 1$.
          
          У всех вершин не может быть степень, равная $1$, ибо тогда отсутствует связность (граф имеет вид пар вершин, инцидентных своему ребру). При этом не может быть и степень, больше либо равная $2$: выберем произвольную вершину и будем просто идти по рёбрам, пока можем. В силу конечности графа мы обязательно придём в вершину, из которой либо нету ребра (то есть её степень равна 1, а такого быть не может), либо мы в ней оказались второй раз и нашли цикл (то есть какие-то 2 вершины соединены не единственным путём), противоречие.
          
          Теперь, доказав наличие вершины степени $1$ в нашем графе, выберем её и рассмотрим граф $G'$ без неё и инцидентного ей ребра. Тогда, к $G'$ применимо предположение индукции и $|V'| = n - 1, |E'| = n - 2$. Для графа $G$ это означает, что $|V| = |V'| + 1 = n, |E| = |E'| + 1 = n - 1$.
        \end{itemize}
      
        \item $3 \Ra 4$ Нужно проверить только ацикличность. Снова воспользуемся индукцией
        \begin{itemize}
          \item База $n = 1$ тривиальна
          
          \item Переход $n > 1$. Предположим, что это не так и есть цикл. Тогда у всех вершин цикла степень $\ge 2$. По лемме о рукопожатиях
          \[
            \suml_{v \in V} \deg v = 2|E| = 2n - 2
          \]
          Отсюда в частности следует, что $\exists v_0 \in V \colon \deg v_0 = 1$, так как если у всех вершин степень $\ge 2$, то сумма степеней $\ge 2n$, а если меньше или равна единице, то сумма $\le n$. Более того, из сказанного выше эта вершина не лежит на цикле. Значит, мы можем её и инцидентное ребро убрать из графа, применить предположение индукции и получить противоречие.
        \end{itemize}
    
        \item $4 \Ra 1$ Снова индукция по $n$ (проблема только со связностью).
      \end{itemize}
    \end{proof}
    
    \Ex
      Пусть $t_n$ - число деревьев на $n$ вершинах. Попробуем заметить некоторую закономерность
      \begin{align*}
        &{t_1 = 1}
        \\
        &{t_2 = 1 = 2^{2 - 2}}
        \\
        &{t_3 = 3 = 3^{3 - 2}}
        \\
        &{t_4 = 16 = 4^{4 - 2}}
        \\
        &{t_5 = 125 = 5^{5 - 2}}
        \\
        &{\vdots}
      \end{align*}
    
    \begin{theorema}{(Формула Кэли, 1857г.)}{}
      \[
        t_n = n^{n - 2}
      \]
    \end{theorema}
    
    \begin{proof}
      Приведём идею с \textit{кодами Прюфера}: из формулы логично предположить, что мы можем каждому дереву на $n$ вершинах сопоставить размещение $n - 2$ чисел из множества $\{1, \ldots, n\}$ с повторениями. Покажем явно алгоритмы, один из которых будет по графу находить код, а другой по нему восстанавливать его.
      \begin{itemize}
        \item Алгоритм, который по графу возвращает код.
        \begin{enumerate}
          \item Выберем вершину степени 1 с наименьшим номером. Допишем справа в уже имеющийся код номер вершины, которая связана с нашей при помощи ребра.
          
          \item Удалим из графа выбранную вершину и инцидентное ей ребро.
          
          \item Повторим итерацию, пока не останется дерево на 2х вершинах.
        \end{enumerate}
        
        \item Алгоритм, который по коду возвращает граф.
        Выпишем последовательность $\{1, \ldots, n\}$, а под ней код, полученный из графа.
        \begin{enumerate}
          \item Выберем самое малое число из верхнего ряда, которого нет в нижнем.
          
          \item Сделаем пару-ребро $(u, v)$, где $u$ - выбранная на предыдущем этапе вершина, $v$ - первая вершина в нижнем ряде.
          
          \item Удалим найденные вершины из рядов.
          
          \item Повторим итерацию, пока не закончится нижний ряд. Сверху останется всего 2 вершины, и они тоже будут образовывать ребро.
        \end{enumerate}
      \end{itemize}
    
      Остаётся обосновать, что полученная функция сопоставления графу его кода - биекция.
      \begin{itemize}
        \item Инъективность
        
        \item Сюръективность
      \end{itemize}
    \end{proof}
    
    \begin{definition}{}{}
      \textit{Унициклическим графом (одноцикловым)} называется связный граф с ровно одним циклом.
    \end{definition}
    
    \begin{note}{}{}
      Из того, что в унициклическом графе всего 1 цикл следует, что этот цикл простой. Более того, унициклический граф на $n$ вершинах - это такой, в котором $|V| = |E| = n$.
    \end{note}
    

      \begin{theorema}{}{}
        $F(n, k) = k \cdot n^{n - 1 - k}$
      \end{theorema}
    
    \begin{note}{}{}
      Если положить за $C(n, n + k)$ - количество связных графов на $n$ вершинах и с $n + k$ рёбрами, то верно следующее:
      \begin{itemize}
        \item \[
          t_n = C(n, n - 1) = n^{n - 2}
        \]
        
        \item \[
          U_n = C(n, n) \sim \sqrt{\frac{\pi}{8}} n^{n - \frac{1}{2}}
        \]
        
        \item \[
          C(n, n + 1) \sim \frac{5}{24}n^{n + 1}
        \]
        
        \item \[
          C(n, n + k) \sim \gamma(k) \cdot n^{n + \frac{3k - 1}{2}}
        \]
      \end{itemize}
    \end{note}
    \section{Деревья. Связь между количеством вершин и рёбер. Эквивалентные определения класса
    деревьев. Формула Кэли для числа деревьев на фиксированном множестве вершин.}
В предыдущем всё
    \section{Кликовое число, число независимости, хроматическое число; связь между этими
    числами. Жадный алгоритм раскраски графа, пример его неоптимальности.}
\end{multicols}