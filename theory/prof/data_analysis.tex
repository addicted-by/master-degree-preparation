\chapter{Основы анализа данных}

\AddToShipoutPictureBG*{%
  \AtPageUpperLeft{%
    \hspace{\paperwidth}%
    \raisebox{-\baselineskip}{%
      \makebox[-10pt][r]{\textbf{ТИИ}}
}}}%

\begin{multicols}{2}
    \raggedcolumns
    \section{Основные понятия машинного обучения. Основные постановки задач. Примеры
    прикладных задач.}

    \section{Линейные пространства. Векторы и матрицы. Линейная независимость. Обратная
    матрица.}

    \section{Производная и градиент функции. Градиентный спуск. Выпуклые функции.}

    \section{Случайные величины. Дискретные и непрерывные распределения. Примеры.}

    \section{Оценивание параметров распределений, метод максимального правдоподобия.
    Бутстрэппинг.}

    \section{Линейные методы классификации и регрессии: функционалы качества, методы
    настройки, особенности применения.}

    \section{Метрики качества алгоритм регрессии и классификации.}

    \section{Оценивание качества алгоритмов. Отложенная выборка, ее недостатки. Оценка полного
    скользящего контроля. Кросс-валидация. Leave-one-out.}

    \section{Деревья решений. Методы построения деревьев. Их регуляризация.}

    \section{Композиции алгоритмов. Разложение ошибки на смещение и разброс.}
    \columnbreak
    \section{Случайный лес, его особенности. Методы поиска выбросов в данных. Методы
    восстановления пропусков в данных. Работа с несбалансированными выборками.}

    \section{Нейронные сети: перцептрон, многослойный перцептрон. Автоэнкодеры и
    рекуррентные нейронные сети.}

    \section{Задача кластеризации. Алгоритм K-Means. Оценки качества кластеризации.
    Литература
    }
\end{multicols}