\chapter{Алгоритмы}

\begin{multicols}{2}
    \raggedcolumns

    \section{Анализ алгоритмов. Понятие о сложности по времени и по памяти. Асимптотика, О-
    символика. Доказательство корректности алгоритмов.}

    \section{Строки и операции над ними. Представление строк. Вычисление длины, конкатенация.
    Алгоритмы поиска подстроки в строке.}

    \section{Сортировки. Нижняя теоретико-информационная оценка сложности задачи сортировки.
    Алгоритмы сортировки вставками, пузырьком, быстрая сортировка, сортировка
    слиянием. Оценка сложности.}

    \section{Представление матриц и векторов. Алгоритмы умножения матриц и эффективные
    способы их реализации. Численные методы решения систем линейных уравнений.}

    \section{Численное дифференцирование и интегрирование. Численные методы для решения
    систем дифференциальных уравнений.}

    \section{Граф. Ориентированный граф. Представления графа. Обход графа в глубину и в ширину.
    Топологическая сортировка. Подсчет числа путей в орграфе.}

    \section{Алгоритмы поиска кратчайших путей в графе. Алгоритм Дейкстры. Алгоритм Форда-
    Беллмана. Алгоритм Флойда. Алгоритм A*.}
    \columnbreak
    \section{Недетерминированные конечные автоматы, различные варианты определения.
    Детерминированные конечные автоматы. Их эквивалентность. Машина Тьюринга.}
\end{multicols}