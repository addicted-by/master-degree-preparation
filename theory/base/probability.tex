\chapter{Теория вероятностей и случайные\\\hfill процессы}

\begin{multicols}{2}
    \raggedcolumns 
    \section{Основные понятия теории вероятностей. Определение вероятностного пространства,
    простейшие дискретные случаи (выборки с порядком и без него, упорядоченные и
    неупорядоченные), классическая вероятностная модель. Случайная величина, функция
    распределения.}
    Математические модели в ТВ. Случайный эксперимент:
    \section{Условные вероятности. Определение условной вероятности, формула полной
    вероятности, формула Байеса.}

    \section{Математическое ожидание, дисперсия, корреляция. Определение математического
    ожидания, дисперсии, ковариации и корреляции, их свойства.}

    \section{Независимость событий. Попарная независимость и независимость в совокупности.}

    \section{Основные теоремы теории вероятностей. Неравенство Чебышева. Закон больших чисел.
    Центральная предельная теорема.
    }

    \section{Распределения. Стандартные дискретные и непрерывные распределения, их
    математические ожидания, дисперсии и свойства: биномиальное, равномерное,
    нормальное, пуассоновское, показательное, геометрическое.}
\end{multicols}